%% LyX 2.1.2 created this file.  For more info, see http://www.lyx.org/.
%% Do not edit unless you really know what you are doing.
\documentclass[doctor,xetex]{thuthesis}
\usepackage{graphicx}

\makeatletter
%%%%%%%%%%%%%%%%%%%%%%%%%%%%%% User specified LaTeX commands.
\input{preamble}

\makeatother

\begin{document}

\chapter{人脸三维模型重建}

本章将着重阐述在已知相机参数条件下,利用多视角图片重建获得人脸三维面片模型。本章重点集中在\ref{sec:=0091CD=005EFA=004E09=007EF4=0070B9=004E91}节,该节详细介绍了通过双目匹配得到三维点云的方法,也是本工作的核心。\ref{sec:=0070B9=004E91=00878D=005408}节将介绍多个点云融合涉及的优化算法,生成mesh模型以及纹理贴图在\ref{sec:Mesh=006A21=00578B=00751F=006210=0053CA=007EB9=007406=008D34=0056FE}节简要介绍。




\section{重建三维点云\label{sec:=0091CD=005EFA=004E09=007EF4=0070B9=004E91}}


\subsection{算法流程}

\begin{figure}[tbph]
\begin{centering}
\includegraphics[width=1\columnwidth]{figures/chpt4/recons}
\par\end{centering}

\protect\caption{算法流程图}
\end{figure}



\subsection{算法模块}


\subsubsection{图片预处理}


\subsubsection{金字塔模型}


\subsubsection{图片对齐\label{sub:=0056FE=007247=005BF9=009F50}}


\subsubsection{NCC匹配计算视差图}


\subsubsection{约束限制}


\paragraph{顺序约束}


\paragraph{平滑约束}


\paragraph{一致性约束}


\subsubsection{重新匹配}


\subsubsection{中值滤波}


\subsubsection{视差图优化}


\subsubsection{从视差图到点云}


\section{点云融合\label{sec:=0070B9=004E91=00878D=005408}}


\section{Mesh模型生成及纹理贴图\label{sec:Mesh=006A21=00578B=00751F=006210=0053CA=007EB9=007406=008D34=0056FE}}


\subsection{泊松重建}


\subsection{表面平滑}


\subsection{纹理贴图}
\end{document}
