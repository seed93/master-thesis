
\ctitle{\thetitle}


\cdegree{工学硕士}


\cdepartment{自动化系}


\cmajor{自动化}


\cauthor{\theauthor}


\csupervisor{刘烨斌\hspace{1.3em}副研究员}


\cdate{二〇一四年十一月}


\etitle{LyX Template for Tsinghua University Thesis}


\edegree{Master of Engineering}


\emajor{Automation}


\eauthor{Liang Ding}


\esupervisor{Associate Professor Liu Yebin}


\edate{November, 2014}
\begin{cabstract}
在计算机视觉领域,与人脸相关的方向包括人脸检测、识别、关键点检测等,其最终的目标都是为了通过人脸识别人物身份。现有的方法和数据库大多局限于二维平面人脸照片,这些照片难以全面地反映一个人的面部特征,因此人脸识别等应用都受到了局限而陷入瓶颈。利用人脸三维模型有望提供更多面部信息,从而突破当前瓶颈。然而现有的人脸三维模型数据库数量有限,表情较少,部分缺少耳朵、下巴等关键部位,因此需要制作一套较丰富完整的人脸三维模型数据库。

本文旨在实现一套人脸三维模型采集和重建系统,同时可以提供在多种光照条件下的不同人物多种表情多角度的二维照片。本工作主要包含三个部分:
\begin{enumerate}
\begin{spacing}{1.5}
\item 搭建了一套多相机、光照可控的硬件系统和并设计完成了配套的同步采集软件,该系统以球形金属杆为框架,配有10台高分辨率彩色相机以及2台kinect二代深度相机,光源包括4盏均匀光照的专业摄影灯以及120个单独可控的LED灯。该系统可以实现多视角变光照同步采集。{\small \par}
\item 本工作提出了一种新的标定装置和算法,可以一次性同时标定多台不同尺度不同视角的相机,与传统的棋盘格标定、一维球体标定等方法相比,具有采集简单、运算速度快、结果准确等显著优势。该标定算法不仅可用于本硬件系统,同样适用各种多相机标定场合,尤其可以解决对不同尺度相机标定的问题。{\small \par}
\item 本文的核心是人脸三维重建算法,该算法基于不同视角的二维图片的逐像素匹配,计算优化视差图,从而鲁棒地重建出高精度的人脸三维点云,并通过一系列的后处理最终获得带有纹理信息的人脸三角面片模型。该算法同样可以用于其他物体的三维重建,有望在文物三维重建等领域发挥作用。{\small \par}\end{spacing}

\end{enumerate}
\end{cabstract}

\ckeywords{人脸;三维;采集;重建;}
\begin{eabstract}
dummy
\end{eabstract}

\ekeywords{face; 3D; capture; reconstruction;}
